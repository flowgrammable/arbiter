\section{Intrinsics}
\label{sec:intrinsics}



\subsection{Data types}
\label{subsec:data_types}

Overload resolution operates over five classes of data:
  \begin{itemize}
    \item
      sequences $\overrightarrow{x}$ -- A finite sequence of values.
    \item
      contexts $ctx$ -- A sequence of variable definitions, function definitions, and assumptions.
    \item
      functions $f$ -- Has an arity, argument and result types, and a constraint.
    \item
      expressions $arg, P, Q$ -- Tree representation of terms and types.
    \item
      integers $i$ -- Integers.
  \end{itemize}



\subsection{Simple operation}
\label{subsec:intrinsics.operations}

Overload resolution relies on several operators and functions. Trival operations, listed below, are considered to cost one time credit.
  \begin{center}
  \begin{tabular}{|c|m{7cm}|}
    \hline
      Operation                                  & Description \\
    \hline
      $x_1 = x_2$ or $x_1 \neq x_2$      & Value equality or inequality \\
    \hline
      for $x \in \overrightarrow{x}$ do $\ldots$ & Single iteration step \\
    \hline
      $f(a_1, \ldots, a_n)$                      & Function call \\
    \hline
      return $x$                                 & Value return \\
    \hline
      if $x$ then $\ldots$ else $\ldots$         & Branch selection \\
    \hline
      $|\overrightarrow{x}|$                     & Sequence of length \\
    \hline
      $\overrightarrow{x}_i$                     & Element selection \\
    \hline
      not $b$                                    & Negation of Boolean value \\
    \hline
      $props(ctx)$                            & Get the assumptions in $ctx$\\
    \hline
      raise ``text''                          & Raise an error \\
    \hline
  \end{tabular}
  \end{center}

Functions contain several pieces of information and are extracted with the following accessor functions for a single time credit.
  \begin{center}
    \begin{tabular}{|c|c|l|}
      \hline
        Operation       & Type          & Description \\
      \hline
        $arity(f)$      & integer       & Number of arguments $f$ accepts \\
      \hline
        $argTypes(f)$   & type sequence & Types of functions arguments \\
      \hline
        $resultType(f)$ & type          & Result type of $f$ \\
      \hline
        $constraint(f)$ & term          & Constraint on arguments of $f$ \\
      \hline
    \end{tabular}
  \end{center}



\subsection{Complex intrinsics}
\label{subsec:intrinsics.complex}

The algorithms rely on a collection of simple operations with non-trivial time complexities.

\paragraph{Structural equivalence.}  Our algorithms often use structural equivalence $e_1 \sim e_2$ which is assumed to have a worst case time complexity of less than $4n$ where $n$ is the maximum number of nodes of either $e_1$ or $e_2$. This is because, in the worst case, $e_1$ and $e_2$ are equivalent and there for each node is visited. However each node is visited twice since it is compared to a node in the other expression and is accessed recursively.

\paragraph{Sequence functions.}  There are a collection of sequence operators used by overload resolution.
  \begin{center}
    \begin{tabular}{|c|c|m{8cm}|}
      \hline
        Operation & Worst-case & Description \\
      \hline
        $\mbox{lookup}(ctx, name)$
          & $2 |ctx|$
          & Collect functions with the name $name$ \\
      \hline
        $\mbox{map}(\overrightarrow{x}, func)$
          & $|\overrightarrow{x}| n$
          & Transform each element of a sequence; $n$ is the worst case
            running-time of $func$ \\
      \hline
        $\mbox{filter}_{\mbox{not}}(\overrightarrow{x}, y)$
          & $2 |\overrightarrow{x}|$
          & Construct a sequence from $\overrightarrow{x}$ without elements that
            are equal to $y$ \\
      \hline
        $\mbox{min}(\overrightarrow{x}, \leq)$
          & $4 n |\overrightarrow{x}|^2$
          & Constructs a sequence of minimal elements from the preorder $\leq$;
            $n$ is the worst-case running time of $\leq$. \\
      \hline
    \end{tabular}
  \end{center}
The min function works by constructing a dependency graph for the elements $\overrightarrow{x}$ and selecting the elements with no in edges.

\paragraph{Substitution.}  Substitution $\mbox{substitute}(\overrightarrow{x},\overrightarrow{y}, e)$ substitutes each instance of $x$ with $y$ in $e$ where $|\overrightarrow{x}| = |\overrightarrow{y}|$. This operation is assumed to take potentially $5 |\overrightarrow{y}| n$ where $n$ is the number of nodes in the expression $e$.



% \subsection{Constants}
% \label{subsec:intrinsics.constants}

% To simplify the discussion of time complexity, we use the following constants